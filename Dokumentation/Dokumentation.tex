\documentclass[11pt,a4paper]{article}
\usepackage[utf8]{inputenc}
\usepackage[ngerman]{babel}
\usepackage{geometry}
\usepackage{amsmath}
\usepackage{outlines}
\geometry{a4paper, margin=2cm}
\usepackage{enumitem}

\title{Dokumentation: Kollisionsdetektion, WPF-Projekt in SSDS}
\author{Ivan Kurilin \and Dimitrij Pivovar}
\date{\today}

\begin{document}
	\maketitle
	\begin{center}
		\textit{Technische Hochschule Köln}
	\end{center}
	\section*{Projektziel}	
	
	Das Ziel dieses WPF-Projekts war es, eine Kollisionsdetektion in einer 3D-Umgebung zu implementieren. Die Objekte sollten eine nahezu realistische Reaktion auf Kollisionen mit den Kugeln darstellen und dem Benutzer verschiedene Methoden zur Veranschaulichung bieten, um die Kollision näher zu betrachten und besser zu verstehen.
	
	\subsection*{Themen im Detail}
	
	\section{Kollisionsdetektion}
	\subsection{Einfluss der Masse auf den Impuls}
	\subsection{Bewegungsvektoren}
	\subsection{Tunneling}
	\subsection{Menü}
	
	
	
	
	
\end{document}










\end{document}
